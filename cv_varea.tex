\documentclass[10pt]{moderncv}
\moderncvtheme[coolblack]{casual}             
\usepackage[utf8]{inputenc}                   
\usepackage[scale=0.8]{geometry}
\definecolor{see}{rgb}{0.5,0.5,0.5}
\definecolor{cb}{rgb}{0.05, 0.44, 0.45}
\recomputelengths                             

\firstname{Agust\'in}
\familyname{Varea}
\title{Desarrolador de Software}
\address{\hspace{.65in}Luciano de Figueroa 568. C\'odigo Postal: X5804BYA. C\'ordoba, Argentina}
\mobile{+54 (0358) 155147517} 
\email{agvarea@gmail.com}      
\photo[64pt]{perfil.jpg}       

\begin{document}
\fontsize{9}{10}
\maketitle
\vspace{-.5cm}

\section{Educaci\'on}
\cventry{2008-2014}{Licenciatura en Ciencias de la Computaci\'on}{\href{www.unrc.edu.ar}{\textcolor{cb}{Universidad Nacional de R\'io Cuarto}}, \href{http://dc.exa.unrc.edu.ar}{\textcolor{cb}{Departamento de Computaci\'on}}}{R\'io Cuarto}{C\'ordoba}
{Cinco a\~nos + tesis.}
\cvline{-----}{\textsc{Cursado terminado.}}	
\vspace{5mm}

\cventry{2008-2012}{Analista en Computaci\'on}{\href{www.unrc.edu.ar}{\textcolor{cb}{Universidad Nacional de R\'io Cuarto}}, \href{http://dc.exa.unrc.edu.ar}{\textcolor{cb}{Departamento de Computaci\'on}}}{R\'io Cuarto}{C\'ordoba}
{Tres a\~nos + tesis.}
\cvline{-----}{\textsc{Tesis}}
\newcounter{foo}
\cvline{T\'itulo}{{\textit{\textcolor{cb}{Suturas en Medicina Veterinaria: Un Prototipo de Desarrollo NCL para la Plataforma de Televisi\'on Digital.}}}}
\label{CE}
\cvline{Directores}{Marcelo Arroyo y Nazareno Aguirre}
\vspace{5mm}

\cventry{2002-2007}{Bachiller orientado en Ciencias Naturales, especialidad: Salud y Ambiente}{Instituto Privado Galileo Galilei}{R\'io Cuarto}{C\'ordoba}{}  


% ---------- EXPERENCIA LABORAL -------------
\section{Experiencias Laborales}

\cventry{ Ascentio }{ Desarrolador }{ noviembre de 2015 a actualidad }{}{}{}
\cvline{}{Herramientas: Java, Spring, Maven, JUnit, MySql, PostgresSQL, AngularJS, Jasmine, Karma, Grunt, Bootstrap, Git, Gitlab, Jira, Puppet, Vagrant, Jenkins }
\vspace{5mm}

\cventry{ Harriague }{ Desarrolador }{ noviembre de 2014 a noviembre de 2015 }{}{}{}
\cvline{}{ Herramientas: Java, Spring, Maven, JUnit, MySql, PostgresSQL, AngularJS, Jasmine, Karma, Grunt, Bootstrap, Git, Gitlab, Jira, Puppet, Vagrant, Jenkins }
\vspace{5mm}

\cventry{ Freelance }{ Desarrolador }{ enero de 2013 a noviembre de 2014 }{}{}{}
\cvline{}{Herramientas: Java, Java Enterprise, Spring, Maven, MySql, JQuery, Bootstrap, Git, Bitbucket, Trello, JUnit, Web Service, Jenkins }
\vspace{5mm}

\cventry{ Freelance }{ Desarrolador }{ abril de 2012 a enero de 2013 }{}{}{}
\cvline{}{Herramientas: Ruby on Rails, PostgresSQL, JQuery, Bootstrap, Git, Github, Pivotal Tracker, Heroku, Cucumber, Capybara } 
\vspace{5mm}


% ---------- EXPERENCIA PROGRAMACION -------------
\newpage
\section{Experiencias en Programaci\'on}

\cvline{ETH Distributed and Outsourced Software Engineering}{Consisti\'o en el desarrollo de una plataforma de juegos de forma distribuida. Se emple\'o la construcci\'on de software basada ​​en contratos en colaboraci\'on con las universidades de diferentes pa\'ises. Fue organizado por la ETH Zurich y dictado por el Prof. Dr. B. Meyer. {\newline Demo: http://www.youtube.com/watch?v=UOLq77YykyA}}
\vspace{5mm}

\cvline{Compilador C--}{Se implement\'o un compilador para un subconjunto de estructuras del lenguaje de programaci\'on C. Se atravezaron las diversas etapas del proceso de compilaci\'on. Se emple\'o Lex && Yacc para el an\'alisis l\'exico y sint\'actico (LALR) respectivamente, c\'odigo tres direcciones como lenguaje intermedio, instrucciones de la familia de CPUs 80x86 para la generaci\'on de c\'odigo para enteros, y por \'ultimo, el co-procesador matem\'atico x87 para la generaci\'on de c\'odigo para punto flotante.}


% ---------- CURSOS -------------
\section{Cursos}

\cventry{2015}{Fundamentos de Lenguajes de Programaci\'on Cu\'antica}{Prof. Alejandro D\'iaz-Caro}{}{}{}
\cvline{}{Veintidosava Escuela de Verano de Ciencias Inform\'aticas.}
\cvline{}{Universidad Nacional de R\'io Cuarto, C\'ordoba, Argentina. Febrero 2015.}

\cventry{2013}{RubyConf Argentina}{}{}{}{}
\cvline{}{Tercera edici\'on de RubyConf Argentina}
\cvline{}{Ciudad Cultural Konex, Buenos Aires, Argentina. Noviembre 2013.}

\cventry{2012}{Mobile Programming With Touchdevelop For Windows Phone}{Dr. Nikolai Tillmann}{}{}{}
\cvline{}{D\'ecimo Novena Escuela de Verano de Ciencias Inform\'aticas.}
\cvline{}{Universidad Nacional de R\'io Cuarto, C\'ordoba, Argentina. Febrero 2012.}

\cventry{2011}{Aprendizaje Autom\'atico usando Kernel Machines: Teor\'ia y Aplicaciones}{Prof. St\'ephane Canu}{}{}{}
\cvline{}{Escuela de Ciencias Inform\'aticas}
\cvline{}{Universidad de Buenos Aires, Bueno Aires, Argentina. Julio 2011.}

\cventry{2011}{Introduction to Mobile Robotic}{Prof. Miroslav Kulich}{}{}{}
\cvline{}{Escuela de Ciencias Inform\'aticas}
\cvline{}{Universidad de Buenos Aires, Bueno Aires, Argentina. Julio 2011.}

\cventry{2011}{“Ingl\'es con Fines Espec\'ificos para las Ciencias Exactas: Desarrollo de Habilidades de Comprensi\'on y Producci\'on Oral del Discurso Acad\'emico”}{}{}{}{}
\cvline{}{Departamento de Computaci\'on.}
\cvline{}{Universidad Nacional de R\'io Cuarto, C\'ordoba, Argentina. Febrero 2011.}

\cventry{2011}{Desarrollo de Aplicaciones NCL para Televisi\'on Digital}{Dr. Federico Balaguer, Leonardo Isasmendi}{}{}{}
\cvline{}{D\'ecima Octava Escuela de Verano de Ciencias Inform\'aticas}
\cvline{}{Universidad Nacional de R\'io Cuarto, C\'ordoba, Argentina. Febrero 2011.}

\cventry{2011}{An\'alisis Autom\'atico de Programas : de la Teor\'ia a la Pr\'actica}{Dr. Diego Garbervetsky, Lic. Guido de Caso}{}{}{}
\cvline{}{D\'ecima Octava Escuela de Verano de Ciencias Inform\'aticas}
\cvline{}{Universidad Nacional de R\'io Cuarto, C\'ordoba, Argentina. Febrero 2011.}

\cventry{2010}{Testing de Aplicaciones Web}{INTI}{}{}{}
\cvline{}{Departamento de Computaci\'on}
\cvline{}{Universidad Nacional de R\'io Cuarto, C\'ordoba, Argentina. Diciembre 2010.}

\cventry{2010}{Verificaci\'on Funcional de Sistemas Digitales Basada en Aserciones (Assertion-Based Verification)}{}{}{}{}
\cvline{}{Escuela de Ciencias Inform\'aticas}
\cvline{}{Universidad de Buenos Aires, Buenos Aires, Argentina. Julio 2010.}

\cventry{2010}{Aprendizaje por Refuerzos: Teor\'ia y Aplicaciones en Rob\'otica, Psicolog\'ia y Neurociencias}{}{}{}{}
\cvline{}{Escuela de Ciencias Inform\'aticas}
\cvline{}{Universidad de Buenos Aires, Buenos Aires, Argentina. Julio 2010.}


% ---------- CONGRESOS -------------
\section{Asistencia a Congresos y Jornadas}

\cventry{2011}{XII Seminario de Cirug\'ia y Taller de Enseñanza de Cirug\'ia: Suturas en Medicina Veterinaria: un Prototipo de Desarrollo NCL para la plataforma de Televisi\'on Digital.}{Martinelli Fern\'an G., Riberi Franco G., Varea Agust\'in}{}{}{}
\cvline{}{Universidad Nacional de Tucum\'an. (San Miguel de Tucum\'an), Tucum\'an, Argentina. Exposici\'on en la modalidad Posters. Octubre 2011.}

\cventry{2011}{40 JAIIO: Suturas en Medicina Veterinaria: un Prototipo de Desarrollo NCL para la plataforma de Televisi\'on Digital.}{Martinelli Fern\'an G., Riberi Franco G., Varea Agust\'in.}{}{}{}
\cvline{}{Universidad Tecnol\'ogica Nacional (UTN), Facultad Regional C\'ordoba, C\'ordoba, Argentina. Agosto 2011.}

\cventry{2011}{PyDay C\'ordoba}{}{}{}{}
\cvline{}{Universidad Tecnol\'ogica Nacional (UTN), Facultad Regional C\'ordoba, C\'ordoba, Argentina. Abril 2011.}

% ---------- Premios -------------
\section{Reconocimientos y Premios}
\cventry{EST 2011}{Segundo Puesto, Categor\'ia: Trabajo Final de Carrera (Simposio EST 2011) }{}{}{}{}
\cvline{}{40 JAIIO.}

% ---------- Publicaciones -------------
\section{Publicaciones}
\cventry{Junio 2012}{Desarrollo De Una Aplicaci\'on de TV Digital Para Enseñanza. Punto Por Punto Suturas.}{Patricia Bertone, Franco Riberi, Fern\'an G. Martinelli, Agust\'in Varea, Carmiña Verde y Marcelo Arroyo.}
{\newline En TE\&ET (VII Congreso Tecnolog\'ia en Educaci\'on y Educaci\'on en Tecnolog\'ia)}{\newline Jun\'in-Pergamino. Buenos Aires. Argentina.}{}

\cventry{Agosto 2011}{Suturas en Medicina Veterinaria: Un Prototipo de Desarrollo NCL para la plataforma de Televisi\'on Digital}{Fern\'an G. Martinelli, Franco G. Riberi y Agust\'in Varea}
{\newline EST 2011 (Simposio Concurso de Trabajos Estudiantiles), 40 JAIIO.}{\newline Universidad Tecnol\'oginca Nacional, C\'ordoba, C\'ordoba, Argentina}{}


% ---------- DOCENCIA -------------
\section{Antecedentes Docentes y de Formaci\'on de Recursos Humanos}

\cventry{2012-2013}{Ayudante de Segunda Rentado}{Departamento de Computaci\'on}{Universidad Nacional de R\'io Cuarto}{R\'io Cuarto, C\'ordoba}{} 
\cvline{}{Periodo: Desde Agosto 2012 hasta Julio 2013.}
\cvline{}{C\'atedra: Ingenier\'ia de Software.}

\cventry{2011-2012}{Ayudante de Segunda Rentado}{Departamento de Computaci\'on}{Universidad Nacional de R\'io Cuarto}{R\'io Cuarto, C\'ordoba}{} 
\cvline{}{Periodo: Desde Agosto 2011 hasta Julio 2012.}
\cvline{}{C\'atedra: An\'alisis y Diseño de Sistemas.}

\cventry{2010-2011}{Ayudante de Segunda Rentado}{Departamento de Computaci\'on}{Universidad Nacional de R\'io Cuarto}{R\'io Cuarto, C\'ordoba}{} 
\cvline{}{Periodo: Desde Agosto 2010 hasta Julio 2011.}
\cvline{}{C\'atedra: Ingenier\'ia de Software.}

\cventry{2009-2010}{Ayudante de Segunda Ad-Honorem }{Departamento de Computaci\'on}{Universidad Nacional de R\'io Cuarto}{R\'io Cuarto, C\'ordoba}{} 
\cvline{}{Periodo: Desde Agosto 2009 hasta Julio 2010.}
\cvline{}{C\'atedra: Introducci\'on a la Algor\'itmica y Programaci\'on.}

% ---------- IDIOMAS -------------
\section{Idiomas}
\cvcomputer{Espa\~nol}{Lengua Nativa}{}{}
\cvcomputer{Ingl\'es}{Competencia Profesional}{}{}

% ---------- APTITUDES -------------
\section{Aptitudes y Conocimientos}
\cvcomputer{S.O}{GNU/Linux, Windows}{}{}
\cvcomputer{Lenguajes}{Java, JavaScript, Ruby, HTML5, Python, CSS3, Less}{}{}
\cvcomputer{Frameworks}{Spring, Rails, AngularJS, ReactJS, Meteor, NodeJS}{}{}
\cvcomputer{Testing}{Selenium, Jasmine, Karma, Cucumber, Capybara, JUnit}{}{}
\cvcomputer{Integraci\'on Continua}{Jenkins}{}{}
\cvcomputer{Base de Datos}{MySQL, PostgreSQL, MongoDB}{}{}
\cvcomputer{Control de Revisiones}{Subversion, Git}{}{}
\cvcomputer{Documentation}{\LaTeXe, Doxygen, Swagger}{}{}


\end{document}