\documentclass[10pt]{moderncv}
\moderncvtheme[coolblack]{casual}             
\usepackage[utf8]{inputenc}                   
\usepackage[scale=0.8]{geometry}
\definecolor{see}{rgb}{0.5,0.5,0.5}
\definecolor{cb}{rgb}{0.05, 0.44, 0.45}
\recomputelengths                             

\firstname{Agustín}
\familyname{Varea}
\title{Computer Scientist}
\address{\hspace{.65in}Luciano de Figueroa 568. C\'odigo Postal: X5804BYA} {C\'ordoba, C\'ordoba, Argentina}
\mobile{+54 (0358) 155147517} 
\email{agvarea@gmail.com}      
\photo[64pt]{perfil.jpg}       

\begin{document}
\fontsize{9}{10}
\maketitle
\vspace{-.5cm}

\section{Educación}
\cventry{2008-2014}{Licenciatura en Ciencias de la Computación}{\href{www.unrc.edu.ar}{\textcolor{cb}{Universidad Nacional de R\'io Cuarto}}, \href{http://dc.exa.unrc.edu.ar}{\textcolor{cb}{Departamento de Computaci\'on}}}{R\'io Cuarto}{C\'ordoba}
{Cinco a\~nos + tesis.}
\cvline{-----}{\textsc{Cursado terminado.}}	

\cventry{2008-2012}{Analista en Computación}{\href{www.unrc.edu.ar}{\textcolor{cb}{Universidad Nacional de R\'io Cuarto}}, \href{http://dc.exa.unrc.edu.ar}{\textcolor{cb}{Departamento de Computaci\'on}}}{R\'io Cuarto}{C\'ordoba}
{Tres a\~nos + tesis.}
\cvline{-----}{\textsc{Tesis}}
\newcounter{foo}
\cvline{Título}{{\textit{\textcolor{cb}{Suturas en Medicina Veterinaria: Un Prototipo de Desarrollo NCL para la Plataforma de Televisi\'on Digital.}}}}
\label{CE}
\cvline{Directores}{Marcelo Arroyo y Nazareno Aguirre}


\cventry{2002-2007}{Bachiller orientado en Ciencias Naturales, especialidad: Salud y Ambiente}{Instituto Privado Galileo Galilei}{R\'io Cuarto}{C\'ordoba}{}  


% ---------- EXPERENCIA LABORAL -------------
\section{Experiencias Laborales}

\cventry{Freelance}{ Desarrolador de Software }{ abril de 2012 a enero de 2013 }{}{}{}
\cvline{}{Herramientas: Java, Java Enterprise, Spring, MySql, Git, Bitbucket, Trello, JUnit, Web Service  
 {\newline Plataforma para la administración de contingencia de barcos. }}

\cventry{Freelance}{Desarrollo de Plataforma Web (SDCA)}{}{}{}{}
\cvline{}{Herramientas: Java, Java Enterprise, Spring, MySql, Git, Bitbucket, Trello, JUnit, Web Service  
 {\newline Plataforma para la administración de contingencia de barcos. }}

\cventry{Freelance}{Desarrollo de Plataforma Web (MSS)}{}{}{}{}
\cvline{}{Herramientas: Java, Java Enterprise, Spring, MySql, Git, Bitbucket, Trello, Heroku, JUnit, Web Service  
 {\newline Plataforma para la administración y sincronización de datos de la plataforma Amazon y Channel Advisor }}

\cventry{Freelance}{Desarrollo de Software Administrativo a Medida (DP)}{}{}{}{}
\cvline{}{Herramientas: Ruby on Rails, PostgresSQL, Git, Github, Pivotal Tracker, Heroku, Cucumber, Capybara. 

\cventry{Freelance}{Desarrollo de Software Administrativo a Medida (Fulbo)}{}{}{}{}
\cvline{}{Herramientas: Ruby on Rails, PostgresSQL, Git, Github, Pivotal Tracker, Heroku, Cucumber, Capybara. 

\cventry{Freelance}{Desarrollo de Software Administrativo a Medida (G-Door)}{}{}{}{}
\cvline{}{Herramientas: Ruby on Rails, PostgresSQL, Git, Github, Pivotal Tracker, Heroku, Cucumber, Capybara. 

\cventry{Freelance}{Desarrollo de Software Administrativo a Medida (GG)}{}{}{}{}
\cvline{}{Herramientas: Ruby on Rails, PostgresSQL, Git, Github, Pivotal Tracker, Heroku, Cucumber, Capybara. 


% ---------- EXPERENCIA PROGRAMACION -------------
\section{Experiencias en Programación}

\cvline{ETH Distributed and Outsourced Software Engineering}{Consistió en el desarrollo de una plataforma de juegos de forma distribuida. Se empleó la construcción de software basada ​​en contratos en colaboración con las universidades de diferentes países. Fue organizado por la ETH Zurich y dictado por el Prof. Dr. B. Meyer. {\newline Demo: http://www.youtube.com/watch?v=UOLq77YykyA}}

\cvline{Compilador C--}{Se implementó un compilador para un subconjunto de estructuras del lenguaje de progracion C. Se atravezaron las diversas etapas del proceso de compilación. Se empleó Lex && Yacc para el análisis l\'exico y sint\'actico (LALR) respectivamente, c\'odigo tres direcciones como lenguaje intermedio, instrucciones de la familia de CPUs 80x86 para la generaci\'on de c\'odigo para enteros, y por \'ultimo, el co-procesador matem\'atico x87 para la generaci\'on de c\'odigo para punto flotante.}


% ---------- CURSOS -------------
\section{Cursos}

\cventry{2012}{Mobile Programming With Touchdevelop For Windows Phone}{Dr. Nikolai Tillmann}{}{}{}
\cvline{}{Décimo Novena Escuela de Verano de Ciencias Informáticas.}
\cvline{}{Universidad Nacional de R\'io Cuarto, C\'ordoba, Argentina. Febrero 2012.}

\cventry{2011}{Aprendizaje Automático usando Kernel Machines: Teoría y Aplicaciones}{Prof. Stéphane Canu}{}{}{}
\cvline{}{Escuela de Ciencias Informáticas}
\cvline{}{Universidad de Buenos Aires, Bueno Aires, Argentina. Julio 2011.}

\cventry{2011}{Introduction to Mobile Robotic}{Prof. Miroslav Kulich}{}{}{}
\cvline{}{Escuela de Ciencias Informáticas}
\cvline{}{Universidad de Buenos Aires, Bueno Aires, Argentina. Julio 2011.}

\cventry{2011}{“Inglés con Fines Específicos para las Ciencias Exactas: Desarrollo de Habilidades de Comprensión y Producción Oral del Discurso Académico”}{}{}{}{}
\cvline{}{Departamento de Computaci\'on.}
\cvline{}{Universidad Nacional de R\'io Cuarto, C\'ordoba, Argentina. Febrero 2011.}

\cventry{2011}{Desarrollo de Aplicaciones NCL para Televisión Digital}{Dr. Federico Balaguer, Leonardo Isasmendi}{}{}{}
\cvline{}{Décima Octava Escuela de Verano de Ciencias Informáticas}
\cvline{}{Universidad Nacional de R\'io Cuarto, C\'ordoba, Argentina. Febrero 2011.}

\cventry{2011}{Análisis Automático de Programas : de la Teoría a la Práctica}{Dr. Diego Garbervetsky, Lic. Guido de Caso}{}{}{}
\cvline{}{Décima Octava Escuela de Verano de Ciencias Informáticas}
\cvline{}{Universidad Nacional de R\'io Cuarto, C\'ordoba, Argentina. Febrero 2011.}

\cventry{2010}{Testing de Aplicaciones Web}{INTI}{}{}{}
\cvline{}{Departamento de Computaci\'on}
\cvline{}{Universidad Nacional de R\'io Cuarto, C\'ordoba, Argentina. Diciembre 2010.}

\cventry{2010}{Verificación Funcional de Sistemas Digitales Basada en Aserciones (Assertion-Based Verification)}{}{}{}{}
\cvline{}{Escuela de Ciencias Informáticas}
\cvline{}{Universidad de Buenos Aires, Buenos Aires, Argentina. Julio 2010.}

\cventry{2010}{Aprendizaje por Refuerzos: Teoría y Aplicaciones en Robótica, Psicología y Neurociencias}{}{}{}{}
\cvline{}{Escuela de Ciencias Informáticas}
\cvline{}{Universidad de Buenos Aires, Buenos Aires, Argentina. Julio 2010.}


% ---------- CONGRESOS -------------
\section{Asistencia a Congresos y Jornadas}

\cventry{2011}{XII Seminario de Cirugía y Taller de Enseñanza de Cirugía: Suturas en Medicina Veterinaria: un Prototipo de Desarrollo NCL para la plataforma de Televisión Digital.}{Martinelli Fernán G., Riberi Franco G., Varea Agustín}{}{}{}
\cvline{}{Universidad Nacional de Tucumán. (San Miguel de Tucumán), Tucumán, Argentina. Exposición en la modalidad Posters. Octubre 2011.}

\cventry{2011}{40 JAIIO: Suturas en Medicina Veterinaria: un Prototipo de Desarrollo NCL para la plataforma de Televisión Digital.}{Martinelli Fernán G., Riberi Franco G., Varea Agustín.}{}{}{}
\cvline{}{Universidad Tecnológica Nacional (UTN), Facultad Regional Córdoba, Córdoba, Argentina. Agosto 2011.}

\cventry{2011}{PyDay Córdoba}{}{}{}{}
\cvline{}{Universidad Tecnológica Nacional (UTN), Facultad Regional Córdoba, Córdoba, Argentina. Abril 2011.}

\section{Premios y Becas Obtenidas}
\cventry{Beca 500x500}{Ministerio de Industria, Comercio y Trabajo (Provincia 
de Córdoba)}{}{}{}{}
\cvline{}{2008-2011 (Período Analista en computación).}

\cventry{Beca 500x500}{Ministerio de Industria, Comercio y Trabajo (Provincia 
de Córdoba)}{}{}{}{}
\cvline{}{2011-2012 (Renovada para licenciatura en Cs. De la Computación).}

\cventry{EST 2011}{Segundo Puesto, Categoría: Trabajo Final de Carrera (Simposio EST 2011) }{}{}{}{}
\cvline{}{40 JAIIO.}

\section{Publicaciones en Congresos Nacionales}
\cventry{Junio 2012}{Desarrollo De Una Aplicación de TV Digital Para Enseñanza. Punto Por Punto Suturas.}{Patricia Bertone, Franco Riberi, Fernán G. Martinelli, Agustín Varea, Carmiña Verde y Marcelo Arroyo.}
{\newline En TE\&ET (VII Congreso Tecnología en Educación y Educación en Tecnología)}{\newline Junín-Pergamino. Buenos Aires. Argentina.}{}

\cventry{Agosto 2011}{Suturas en Medicina Veterinaria: Un Prototipo de Desarrollo NCL para la plataforma de Televisión Digital}{Fernán G. Martinelli, Franco G. Riberi y Agustín Varea}
{\newline EST 2011 (Simposio Concurso de Trabajos Estudiantiles), 40 JAIIO.}{\newline Universidad Tecnol\'oginca Nacional, C\'ordoba, C\'ordoba, Argentina}{}


% ---------- DOCENCIA -------------
\section{Antecedentes Docentes y de Formación de Recursos Humanos}

\cventry{2012-2013}{Ayudante de Segunda Rentado}{Departamento de Computaci\'on}{Universidad Nacional de R\'io Cuarto}{R\'io Cuarto, C\'ordoba}{} 
\cvline{}{Periodo: Desde Agosto 2012 hasta Julio 2013.}
\cvline{}{Cátedra: Ingeniería de Software.}

\cventry{2011-2012}{Ayudante de Segunda Rentado}{Departamento de Computaci\'on}{Universidad Nacional de R\'io Cuarto}{R\'io Cuarto, C\'ordoba}{} 
\cvline{}{Periodo: Desde Agosto 2011 hasta Julio 2012.}
\cvline{}{Cátedra: Análisis y Diseño de Sistemas.}

\cventry{2010-2011}{Ayudante de Segunda Rentado}{Departamento de Computaci\'on}{Universidad Nacional de R\'io Cuarto}{R\'io Cuarto, C\'ordoba}{} 
\cvline{}{Periodo: Desde Agosto 2010 hasta Julio 2011.}
\cvline{}{Cátedra: Ingeniería de Software.}

\cventry{2009-2010}{Ayudante de Segunda Ad-Honorem }{Departamento de Computaci\'on}{Universidad Nacional de R\'io Cuarto}{R\'io Cuarto, C\'ordoba}{} 
\cvline{}{Periodo: Desde Agosto 2009 hasta Julio 2010.}
\cvline{}{Cátedra: Introducción a la Algorítmica y Programación.}

\section{Posiciones Administrativas}
\cventry{2009-2010}{Miembro de la Comisión Curricular, Departamento de Computación de las carrerras de profesorado en ciencias de la computación, licenciatura en ciencias de la computación y analista en ciencias de la computación
}{Universidad Nacional R\'io Cuarto}{C\'ordoba, Argentina}{}{}


% ---------- APTITUDES -------------
\section{Aptitudes y Conocimientos}
\cvcomputer{S.O}{GNU/Linux, Windows}{}{}
\cvcomputer{Lenguajes}{Java, Pascal, Haskell, Ruby, HTML5, Python, JS, CSS3, JavaScript}{}{}
\cvcomputer{Frameworks}{Spring, Rails, AngularJS, MeanIO, NodeJS}{}{}
\cvcomputer{Testing}{Cucumber, Capybara, JUnit}{}{}
\cvcomputer{Base de Datos}{MySQL, PostgreSQL}{}{}
\cvcomputer{Control de Revisiones}{Subversion, Git}{}{}
\cvcomputer{Documentation}{\LaTeXe, Doxygen}{}{}
\cvcomputer{IDE's}{Eclipse, NetBeans, EiffelStudio}{}{}


\end{document}